\documentclass{beamer}
\mode<presentation>{
  \usetheme{Warsaw}

  \definecolor{beamer@blendedred}{rgb}{0.7,0.1,0.1}
  \setbeamercolor{structure}{fg=beamer@blendedred}

  \setbeamercovered{invisible}
}

\usepackage[T2A]{fontenc}
\usepackage[utf8]{inputenc}
\usepackage[russian]{babel}
\usepackage{multimedia}
\usepackage{hyperref}
\logo{\includegraphics[height=0.3cm]{valknut.png}}

\definecolor{ExclamMark}{RGB}{34, 89, 150}

\title{GO!!}
\author{Максим Трескин\\ \texttt{mtreskin@metachord.com} \\ \texttt{@mtreskin}}
\date[2011.06.25]{25 июня, 2011}

\begin{document}

\begin{frame}
  \titlepage
\end{frame}


\begin{frame}
  \frametitle{GO!!}
  \begin{minipage}[t]{90px}
    \hfil
    \includegraphics[height=7cm]{gopher_c.png}
    \hfil
  \end{minipage}
\end{frame}

\begin{frame}
  \frametitle{Кто и когда?}
  Ken Thompson, Rob Pike и Robert Griesemer @ Google\\
  21 Сентября 2007 года
\end{frame}

\begin{frame}
  \frametitle{Зачем?}
  \begin{itemize}
  \item Ускорение разработки приложений
  \item Решение проблемы зависимостей
  \item Избавление от громоздких систем типов (C++, Java), повышение динамичности
  \item Хорошая поддержка сборки мусора и параллельных вычислений
  \item Утилизация ресурсов многоядерных систем
  \end{itemize}
\end{frame}


\begin{frame}
  \frametitle{Что?}
  \begin{itemize}
  \item Язык общего назначения, особенно пригодный для системного программирования
  \item Строго типизирован, со сборкой мусора и поддержкой многопоточности
  \item Приложения составляются из пакетов
  \item Очень компактная грамматика (полезно для реализации инструментов, работающих с кодом)
  \end{itemize}
\end{frame}



\begin{frame}[fragile]
  \frametitle{Методы могут быть определены для любого типа данных}
\begin{verbatim}
type Complex struct {
    real    float64
    imag    float64
}

func (cn *Complex) modulus() float64 {
    sum := math.Pow(cn.real, 2) + math.Pow(cn.imag, 2)
    return math.Sqrt(sum)
}

\end{verbatim}
\end{frame}


\begin{frame}[fragile]
  \frametitle{Нет наследования, есть интерфейсы}
\begin{verbatim}
type Client struct {
    id      int
    name    string
}

type Manager struct {
    name    string
    clients  []int
}
\end{verbatim}
\end{frame}


\begin{frame}[fragile]
  \frametitle{Нет наследования, есть интерфейсы}
\begin{verbatim}
type Printable interface {
    Print()
}

func (cl Client) Print() {
    fmt.Printf("ID: %d, Name: %s\n", cl.id, cl.name)
}

func (mg Manager) Print() {
    fmt.Printf("Name: %s, Clients: %d\n",
               mg.name, len(mg.clients))
}
\end{verbatim}
\end{frame}


\begin{frame}[fragile]
  \frametitle{Нет наследования, есть интерфейсы}
\begin{verbatim}
func main() {
    var c *Client
    var m *Manager
    var p Printable

    c = new(Client); c.id = 5; c.name = "Kosh"
    m = new(Manager); m.name = "Krabutnica"; m.level = 10

    p = c; p.Print()
    p = m; p.Print()
}
\end{verbatim}
\end{frame}


\begin{frame}
  \frametitle{Кое-что, облегчающее жизнь}
  \begin{itemize}
  \item Функции — значения первого класса
    \pause
  \item Функции могут возвращать несколько значений
    \pause
  \item Нет перегрузки операторов или методов
    \pause
  \item Нет неявного преобразования численных типов
  \end{itemize}
\end{frame}



\begin{frame}
  \frametitle{Многопоточность}
  \Large{\textit{\color{ExclamMark}{Не общайтесь с помощью раздельной памяти. Разделяйте память с помощью общения}}}
\end{frame}

\begin{frame}
  \frametitle{Многопоточность}
  \begin{itemize}
  \item Goroutines. Just add \texttt{go}
    \pause
  \item Каналы
  \end{itemize}
\end{frame}


\begin{frame}[fragile]
  \frametitle{Многопоточность: Каналы}
\begin{verbatim}
//Создание
КаналЦелых := make(chan int)
КаналФайлов := make(chan *os.File, 100)

//Посылка
go func() {
    КаналЦелых <- 14
}()

//Приём
ЕстьЧто := <- КаналЦелых
\end{verbatim}
\end{frame}


\begin{frame}[fragile]
  \frametitle{Обработка ошибок (и не только): Defer}
\begin{verbatim}
func Contents(filename string) (string, os.Error) {
    f, err := os.Open(filename, os.O_RDONLY, 0)
    if err != nil {
        return "", err
    }
    defer f.Close()
    ....
}
\end{verbatim}
\end{frame}


\begin{frame}
  \frametitle{Ещё немного хорошего}
  \begin{itemize}
  \item Слайсы — ссылки на куски массивов, способные динамически изменять размер
  \item Ассоциативные массивы
  \item Форматирование произвольных структур (\texttt{"\%v"})
  \end{itemize}
\end{frame}

\begin{frame}
  \frametitle{Инфраструктура}
  \begin{itemize}
  \item godoc — документирование
  \item gofmt — форматирование кода (аналог всяких разных *tidy или indent для Go)
  \item gotest — легковесный фреймворк для тестирования
  \item gofix — обновление кода для новых версий языка
  \item goinstall — установка библиотек
  \end{itemize}
\end{frame}

\begin{frame}
  \frametitle{Ресурсы}
  \begin{itemize}
  \item {\color{blue}\url{http://golang.org}}
  \item golang-nuts, golang-russian на гуглогруппах
  \item GolangRu на Фейсбуці
  \end{itemize}
\end{frame}


\begin{frame}
  \frametitle{Вопросы?}
  \titlepage
\end{frame}


\end{document}
